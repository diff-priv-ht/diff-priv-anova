
\section{Appendix: Direct Calculation of $\sigma$}
\label{Sec:AppDirSig}

In our work, we used $\widehat{SE}$ to form an estimator for $\sigma$ when calculating the null distribution.  We developed another version of a differentially-private ANOVA that calculates $\sigma$ directly using a portion of the privacy budget.  We first define a generalized variance and prove the sensitivity bounds on this quantity.

\begin{definition}[$\var_q$] \label{def:varq} Given a database \x with $k$ groups and $n_j$ entries in the $j$-th group, the $\var_q$ calculation is defined as
\begin{equation*}
\var_q = \sum_{i=1}^N \lvert y_{i} - \overline{y} \rvert^q
\end{equation*}
where $q$ is a positive real number.
\end{definition}

\begin{theorem}[\varq-Sensitivity] \label{thm:varqSens} 
The sensitivity of \varq is bounded above by
$$ \frac{N-1}{N^q} + 1 $$
when $q<1$, and is bounded above by
$$ (N-1)(1-(1-1/N)^q) + 1 $$
when $q>1$. Note that these both give a bound of $2-1/N$ when $q=1$.
\end{theorem}

\begin{proof}
Consider databases \x and \xprime which differ in entry $r$. Recall that the sensitivity of the grand mean is $1/N$. Let $t_i = \left\vert  y_i - \grand \right\vert ^q$. When $q<1, t_i$ is a concave function with positive range, so the worst case sensitivity is between $ y_i - \grand = 0$ and $ y_i - \grand = 1/N$. That is, 
\begin{align*}
\Delta t_i &= \left\vert t_{i}(0) - t_i\left(\frac{1}{N}\right) \right\vert \\
	&= \left( \frac{1}{N} \right)^q.
\end{align*}

Every single term can be affected by at most $\left( \frac{1}{N} \right)^q$, except the term $r$, which can change by $1$. So, 
\begin{align*}
\Delta \varq &\le \left\vert (N-1)(\frac{1}{N})^q + 1 \right\vert \\
	&= \frac{N-1}{N^q} + 1.
\end{align*}
%
When $q>1, t_i$ is convex. The worst case sensitivity is between  $y_i - \grand = 1$ and $y_i - \grand = 1-1/N$ Then,
\begin{align*}
\Delta \varq \le \left\vert (N-1)(1-(1-\frac{1}{N})^q) + 1 \right\vert.
\end{align*}
\end{proof}

We again use the Laplace mechanism, and the sensitivity of a database's variance follows directly from Thm.~\ref{thm:varqSens}:

\begin{corollary}
\label{thm:varsens}
The sensitivity of the variance of a database is bounded above by
$$ 3 + 1/N^2 - 3/N.$$
\end{corollary}

Algorithm~\ref{alg:F1hatVar} divides the privacy budget into $\rho_1$ for the \sa, $\rho_2$ for the \se, and $\rho_3$ for the $\var$ calculations respectively.  The values of $\rho$ are provided as input along with the database \x and the $\eps$ value. When $\widehat{SE}$ is used as the $\sigma$ estimate, we found a 70-30 split of the privacy budget between \sa and \se was optimal (Figure~\ref{Fig:f1-epsfrac}).   In the power analysis of Algorithm~\ref{alg:F1hatVar}, we vary the proportion of the privacy budget dedicated to the \var calculation.  We fixed the proportion of the privacy budget for $\rho_1$ and $\rho_2$ to be a 70-30 split budget not used by $\rho_3$ (the \var calcuation).  

\begin{algorithm}
    \begin{algorithmic}
        \STATE \textbf{Input:} Database \x, $\eps$ value, $\rho_1, \rho_2, \rho_3$
        \STATE Compute $\widehat{\sa} = \sa + Z_1$ where $Z_1\sim \lap\left(\frac{4}{\eps \rho_1}\right)$
        \STATE Compute $\widehat{\se} = \se+ Z_2$ where $Z_2\sim \lap\left(\frac{3}{\eps \rho_2}\right)$
        \STATE Compute $\!\!\widehat{\var} \!\!= \!\!\var \!\!+\! Z_3\!$ where  $\!Z_3 \!\sim\! \lap\!\!\left(\! \frac{3\!+\!1/N^2\! - \!3/N}{\eps\rho_3} \!\right)$
        \STATE Compute $\widehat{F_1} = \frac{\widehat{\sa}/(\k-1)}{\widehat{\se}/(\dbsize-\k)}$
        \STATE \textbf{Output:} $\widehat{F_1}, \widehat{\sa}, \widehat{\se}, \widehat{\var}$
    \end{algorithmic}
    \caption{Differentially private $F_1$-statistic with direct calculation of variance}
     \label{alg:F1hatVar}
\end{algorithm}

Unsurprisingly, allocating part of the budget to calculating the standard deviation negatively impacts the statistical power of the test (Fig.~\ref{Fig:multiple-rhos}). However, it is surprising that this allocation does not impact the power of the test more strongly, since \var has a larger sensitivity and requires adding noise to smaller values than the \sa and \se. 

\begin{figure}[h]
\centering
\includegraphics[width=\linewidth]{multiple-rhos.png}
\caption{The power of $F_1$ for a fixed $\epsilon$ value budgeted across \sa, \se, and \var calculations, with the portion of $\epsilon$ not allocated to $\rho_3$ allocated between $\rho_1$ and $\rho_2$ with a 70-30 split.\label{Fig:multiple-rhos}}
\end{figure}
